\begin{enumerate}
\item \textbf{Улучшение работы с некратными матрицами:}  \\
   Для матриц, размерности которых не кратны степени двойки, можно использовать методы, минимизирующие влияние дополнений:
   \begin{itemize}
       \item Применять блочное разбиение матриц для вычислений только с реальными данными.
       \item Уменьшать размер дополнений за счет использования гибридных методов умножения.
   \end{itemize}

\item \textbf{Оптимизация параллельной версии: } 
   \begin{itemize}
       \item Распределение задач между процессами с учетом их загрузки. Например, процессы с более высокой нагрузкой могут обрабатывать меньшие подзадачи.
       \item Использование асинхронной передачи данных для снижения накладных расходов на синхронизацию.
   \end{itemize}

\item \textbf{Проведение дополнительных тестов:} 
   \begin{itemize}
       \item Проверка алгоритма на матрицах с некратными степеням двойки размерностями.
       \item Анализ влияния изменения порогового значения для перехода к стандартному умножению (вместо рекурсии) на производительность.
   \end{itemize}
\end{enumerate}

