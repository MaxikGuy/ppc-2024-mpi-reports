\subsection{Условия эксперимента}
Для проверки эффективности последовательной и параллельной реализаций алгоритма Штрассена были проведены вычисления на случайных матрицах размером \(128 \times 128\), \(256 \times 256\), \(512 \times 512\) и \(1024 \times 1024\).  

Эксперименты проводились с использованием разного числа процессов (\(1\), \(2\), \(4\), \(7\)). Для каждой комбинации числа процессов и размера матрицы программа запускалась \(5\) раз, а измеренное время усреднялось.  

Важно отметить, что алгоритм эффективен для матриц, размерность которых является степенью двойки. Если размерность матрицы не соответствует этому условию, она автоматически дополняется до ближайшей степени двойки. Например, матрица размером \(600 \times 600\) дополняется нулями до размера \(1024 \times 1024\), что приводит к дополнительным накладным вычислениям.


\subsection{Результаты экспериментов}
Результаты экспериментов представлены в таблице (время указано в секундах):

\begin{table}[H]
\begin{tabular}{|c|c|c|c|c|}
\hline
Число процессов & 128 × 128 & 256 × 256 & 512 × 256 & 1024 × 1024 \\ \hline
1               & 0.11      & 0.83      & 5.76      & 41.96       \\ \hline
2               & 0.09      & 0.54      & 4.14      & 25.43       \\ \hline
4               & 0.06      & 0.48      & 2.75      & 18.85       \\ \hline
7               & 0.06      & 0.41      & 2.36      & 14.90       \\ \hline
\end{tabular}
\end{table}


\subsection{Анализ результатов}
\begin{enumerate}
\item \textbf{Ускорение при увеличении числа процессов:} 
   \begin{itemize}
       \item На больших размерах матриц (\(512 \times 512\) и \(1024 \times 1024\)) наблюдается значительное сокращение времени вычислений при увеличении числа процессов. Например, для матриц размером \(1024 \times 1024\) переход с \(1\) процесса на \(7\) уменьшает время выполнения с \(41.96\) секунд до \(14.90\) секунд.
       \item Для меньших матриц (\(128 \times 128\)) выигрыш во времени минимален, так как накладные расходы на синхронизацию и передачу данных перекрывают эффект параллелизации.
   \end{itemize}

\item\textbf{Эффективность при кратных степеням двойки матрицах:}  \\
   Алгоритм показывает максимальную производительность для матриц, размерность которых кратна степени двойки. Это связано с особенностью разбиения матриц в методе Штрассена.  

   Однако для некратных степеням двойки матриц происходит дополнение до ближайшей большей размерности, что увеличивает объем вычислений. Например, при умножении матриц размером \(600 \times 600\) они будут дополнены до \(1024 \times 1024\). Это приводит к значительному увеличению времени выполнения и требует дополнительных ресурсов.

\item \textbf{Пределы эффективности:}\\
   При увеличении числа процессов до \(4\) или \(7\) наблюдается уменьшение времени выполнения, но эффект не линейный. Это связано с:
   \begin{itemize}
       \item Накладными расходами на синхронизацию процессов.
       \item Нехваткой достаточного количества задач для равномерного распределения между процессами при малых размерах матриц.
   \end{itemize}
\end{enumerate}
